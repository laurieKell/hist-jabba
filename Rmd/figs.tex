% Options for packages loaded elsewhere
\PassOptionsToPackage{unicode}{hyperref}
\PassOptionsToPackage{hyphens}{url}
\documentclass[
]{article}
\usepackage{xcolor}
\usepackage[margin=1in]{geometry}
\usepackage{amsmath,amssymb}
\setcounter{secnumdepth}{5}
\usepackage{iftex}
\ifPDFTeX
  \usepackage[T1]{fontenc}
  \usepackage[utf8]{inputenc}
  \usepackage{textcomp} % provide euro and other symbols
\else % if luatex or xetex
  \usepackage{unicode-math} % this also loads fontspec
  \defaultfontfeatures{Scale=MatchLowercase}
  \defaultfontfeatures[\rmfamily]{Ligatures=TeX,Scale=1}
\fi
\usepackage{lmodern}
\ifPDFTeX\else
  % xetex/luatex font selection
\fi
% Use upquote if available, for straight quotes in verbatim environments
\IfFileExists{upquote.sty}{\usepackage{upquote}}{}
\IfFileExists{microtype.sty}{% use microtype if available
  \usepackage[]{microtype}
  \UseMicrotypeSet[protrusion]{basicmath} % disable protrusion for tt fonts
}{}
\makeatletter
\@ifundefined{KOMAClassName}{% if non-KOMA class
  \IfFileExists{parskip.sty}{%
    \usepackage{parskip}
  }{% else
    \setlength{\parindent}{0pt}
    \setlength{\parskip}{6pt plus 2pt minus 1pt}}
}{% if KOMA class
  \KOMAoptions{parskip=half}}
\makeatother
\usepackage{graphicx}
\makeatletter
\newsavebox\pandoc@box
\newcommand*\pandocbounded[1]{% scales image to fit in text height/width
  \sbox\pandoc@box{#1}%
  \Gscale@div\@tempa{\textheight}{\dimexpr\ht\pandoc@box+\dp\pandoc@box\relax}%
  \Gscale@div\@tempb{\linewidth}{\wd\pandoc@box}%
  \ifdim\@tempb\p@<\@tempa\p@\let\@tempa\@tempb\fi% select the smaller of both
  \ifdim\@tempa\p@<\p@\scalebox{\@tempa}{\usebox\pandoc@box}%
  \else\usebox{\pandoc@box}%
  \fi%
}
% Set default figure placement to htbp
\def\fps@figure{htbp}
\makeatother
\setlength{\emergencystretch}{3em} % prevent overfull lines
\providecommand{\tightlist}{%
  \setlength{\itemsep}{0pt}\setlength{\parskip}{0pt}}
\usepackage{booktabs}
\usepackage{longtable}
\usepackage{array}
\usepackage{multirow}
\usepackage{wrapfig}
\usepackage{float}
\usepackage{colortbl}
\usepackage{pdflscape}
\usepackage{tabu}
\usepackage{threeparttable}
\usepackage{threeparttablex}
\usepackage[normalem]{ulem}
\usepackage{makecell}
\usepackage{xcolor}
\usepackage{bookmark}
\IfFileExists{xurl.sty}{\usepackage{xurl}}{} % add URL line breaks if available
\urlstyle{same}
\hypersetup{
  pdftitle={Final Figs},
  hidelinks,
  pdfcreator={LaTeX via pandoc}}

\title{Final Figs}
\author{}
\date{\vspace{-2.5em}}

\begin{document}
\maketitle

{
\setcounter{tocdepth}{2}
\tableofcontents
}
\subsection{Post processing of JABBA}\label{post-processing-of-jabba}

\subsection{Figures}\label{figures}

\begin{center}\includegraphics{figs_files/figure-latex/fig1-1} \end{center}

\textbf{Figure 1.} Reconstructed time series of ICES historical catches
by stock. The blue line represents catches used in the official stock
assessment while the magenta line is the historical catches as derived
from the ICES historical catch database. Catches are standardised to
have a mean of 0 and a standard deviation of 1 and the y-axis labels are
removed for an easier presentation of the database.

\begin{center}\includegraphics{figs_files/figure-latex/fig2-1} \end{center}

\textbf{Figure 2.} Density plot of the posteriors of the shape parameter
of the production function (BMSY/SSB0) estimated for each stock with
``ICES catches JABBA'' and ``Historical catches JABBA'' models.

\begin{center}\includegraphics{figs_files/figure-latex/fig3-1} \end{center}

\textbf{Figure 3.} Density plot of the posteriors of the population
growth parameter (r) estimated for each stock with ``ICES catches
JABBA'' and ``Historical catches JABBA'' models. The black vertical line
indicates the r prior value used for the analysis.

\begin{center}\includegraphics{figs_files/figure-latex/fig4-1} \end{center}

\textbf{Figure 4.} Density plot of the posteriors of the ratio between
\(B_{current}\) and \(B_{MSY}\) estimated for each stock with ``ICES
catches JABBA'' and ``Historical catches JABBA'' models. The black
vertical line indicates BMSY for each stock.

\begin{center}\includegraphics{figs_files/figure-latex/fig5-1} \end{center}

\textbf{Figure 5.} Trend in stock status (B/BMSY) for each stock
estimated by ``ICES catches JABBA'' and ``Historical catches JABBA''
models. The horizontal dash line indicates BMSY.

\begin{center}\includegraphics{figs_files/figure-latex/fig6a-1} \end{center}

\url{http://127.0.0.1:37757/graphics/998e2f96-10ef-44f8-b4b0-0b654c030dca.png}

\textbf{Figure 6.} Density plot of fishing mortality (median
\(F/F_{MSY}\)) estimated by ``ICES catches JABBA'' and ``Historical
catches JABBA'' models.

\begin{center}\includegraphics{figs_files/figure-latex/fig6b-1} \end{center}

\textbf{Figure 6.} Density plot of the aggregated stock status (median
\(B/B_{MSY}\)) estimated by ``ICES catches JABBA'' and ``Historical
catches JABBA'' models.

\textbf{Figure 6.} Density plot of the aggregated stock status (median
\(B/B_{MSY}\)) (a) and fishing mortality (median \(F/F_{MSY}\)) (b)
estimated by ``ICES catches JABBA'' and ``Historical catches JABBA''
models.

Figure 7 illustrates the estimates across stocks of r, shape (i.e.,
BMSY/K), \(K\), the spawning biomass estimated at the start of the
current stock assessment model (\(B_{initial}\)) and the spawning
biomass estimated at the terminal year of the current stock assessment
model (\(B_{current}\)) from both model scenarios. The results confirm
that the largest difference when including historical catch data is
observed for the shape of the production function and the spawning
biomass estimated at the start of the current stock assessment model
while differences are smaller for r, SSB0 and Bcurrent.

\begin{center}\includegraphics{figs_files/figure-latex/fig7-1} \end{center}

\textbf{Figure 7.} Density of the aggregated posterior distribution for
the population growth parameter (r), shape (i.e., BMSY/SSB0), the
spawning biomass estimated at the start of the ICES times series
(Binitial) and the spawning biomass estimated at the end of the ICES
times series (Bcurrent; in log scale), the stock carrying capacity
(SSB0; in log scale) for both scenarios, ``ICES catches JABBA'' and the
``Historical catches JABBA'' models.

\end{document}
